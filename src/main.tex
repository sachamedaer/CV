%%%%%%%%%%%%%%%%%
% This is an sample CV template created using altacv.cls
% (v1.1.2, 1 February 2017) written by LianTze Lim (liantze@gmail.com). Now compiles with pdfLaTeX, XeLaTeX and LuaLaTeX.
% 
%% It may be distributed and/or modified under the
%% conditions of the LaTeX Project Public License, either version 1.3
%% of this license or (at your option) any later version.
%% The latest version of this license is in
%%    http://www.latex-project.org/lppl.txt
%% and version 1.3 or later is part of all distributions of LaTeX
%% version 2003/12/01 or later.
%%%%%%%%%%%%%%%%

%% If you need to pass whatever options to xcolor
\PassOptionsToPackage{dvipsnames}{xcolor}

%% If you are using \orcid or academicons
%% icons, make sure you have the academicons 
%% option here, and compile with XeLaTeX
%% or LuaLaTeX.
% \documentclass[10pt,a4paper,academicons]{altacv}
\documentclass[9pt,a4paper]{altacv}

%% AltaCV uses the fontawesome and academicon fonts
%% and packages. 
%% See texdoc.net/pkg/fontawecome and http://texdoc.net/pkg/academicons for full list of symbols.
%% 
%% Compile with LuaLaTeX for best results. If you
%% want to use XeLaTeX, you may need to install
%% Academicons.ttf in your operating system's font 
%% folder.


% Change the page layout if you need to
\geometry{left=1cm,right=9cm,marginparwidth=6.8cm,marginparsep=1.2cm,top=1.25cm,bottom=1.25cm}

% Change the font if you want to.

% If using pdflatex:
\usepackage[utf8]{inputenc}
\usepackage[T1]{fontenc}
\usepackage[default]{lato}

\usepackage{hyperref}
\hypersetup{
    colorlinks=true,
    linkcolor=blue,
    filecolor=magenta,      
    urlcolor=cyan,
}

% If using xelatex or lualatex:
% \setmainfont{Lato}

% Change the colours if you want to
\iffalse
\definecolor{Mulberry}{HTML}{72243D}
\definecolor{SlateGrey}{HTML}{2E2E2E}
\definecolor{LightGrey}{HTML}{666666}
\colorlet{heading}{Sepia}
\colorlet{accent}{Mulberry}
\colorlet{emphasis}{SlateGrey}
\colorlet{body}{LightGrey}
\fi

\definecolor{accent}{HTML}{000e17}
\definecolor{heading}{HTML}{000e17}
\definecolor{emphasis}{HTML}{696969}
\definecolor{body}{HTML}{01415f}

\colorlet{heading}{heading}
\colorlet{accent}{accent}
\colorlet{emphasis}{emphasis}
\colorlet{body}{body}

% Change the bullets for itemize and rating marker
% for \cvskill if you want to
\renewcommand{\itemmarker}{{\small\textbullet}}
\renewcommand{\ratingmarker}{\faCircle}

%% sample.bib contains your publications
\addbibresource{sample.bib}

\begin{document}
\name{Sacha Medaer}
\tagline{Looking for a PhD at the intersection of Computer Sciences and Physics Engineering}
%\photo{2.8cm}{Globe_High}
\personalinfo{%
  % Not all of these are required!
  % You can add your own with \printinfo{symbol}{detail}
  \email{sacha@medaer.me}
  \phone{+32471048908}
  \location{Brussels}
  \mailaddress{Europalaan 13, Hoeilaart 1560, Belgium\hspace{10em}}
  \linkedin{\href{https://www.linkedin.com/in/sacha-medaer/}{linkedin.com/in/sacha-medaer/}}
  \github{\href{https://github.com/sachamedaer}{github.com/sachamedaer/}}
  %% You MUST add the academicons option to \documentclass, then compile with LuaLaTeX or XeLaTeX, if you want to use \orcid or other academicons commands.
%   \orcid{orcid.org/0000-0000-0000-0000}
}

%% Make the header extend all the way to the right, if you want. Extend the right margin by 8cm (=6.8cm marginparwidth + 1.2cm marginparsep)
\begin{adjustwidth}{}{-8cm}
\makecvheader
\end{adjustwidth}

%% Provide the file name containing the sidebar contents as an optional parameter to \cvsection.
%% You can always just use \marginpar{...} if you do
%% not need to align the top of the contents to any
%% \cvsection title in the "main" bar.

\cvsection[page1sidebar]{Education}
% for converting grade, see : https://www.jobline.uni-muenchen.de/app_language/degrees_and_grades/grades/index.html

\cvevent{Master degree in Advanced Optical Technologies}{Friedrich-Alexander Universit\"at}{2020}{Nuremberg, Germany}
\begin{itemize}
\small
\item Majors: Computational Optics \& Optics in Communication and Information Theory
\item Final grade: First-class 
\end{itemize}
%\divider

\cvevent{Master degree in Computational Engineering}{Friedrich-Alexander Universit\"at}{2020}{Nuremberg, Germany}
\begin{itemize}
\small
\item Majors: Digital Communication \& Machine Learning
\item Final grade: Upper second-class (2:1)
\end{itemize}
%\divider


\cvevent{Bachelor degree in Physics Engineering}{Université Libre de Bruxelles}{2017}{Brussels, Belgium}

%\divider

\cvevent{Bachelor degree in Computer Sciences}{Université Libre de Bruxelles}{2017}{Brussels, Belgium}

%\divider


\cvevent{World Tour - Work \& Travel Visa}{\vspace{-1.3em}}{2013}{USA, Australia, Thailand}

%\divider

\cvevent{High-school Degree}{\vspace{-1.3em}}{2012}{Nivelles, Belgium}


\vspace{1.5em}
\cvsection[page1sidebar]{Experience and projects}

\cvevent{Deep Reinforcement Learning for the Planning of Liver Tumor Thermal Ablation}{Siemens Healthineers - Artificial Intelligence Group}{2020}{Erlangen, Germany}

\cvevent{Simulation of a Figure-of-eight All-normal-dispersion All-in-fiber Regenerative Hybrid Harmonic Mode-locked Laser}{Max Planck Institute for the Science of Light - Pr. Schmauss Associated Group}{2020}{Erlangen, Germany}

\cvevent{Optcom: Open-Source Optical System Simulator (\href{https://github.com/optcom-org/optcom}{Github})}{Founder}{2019}{Germany}


\cvevent{Simulation of a Pulse Stretcher concept for Ultrashort Pulse Fiber Laser Systems}{Max Planck Institute for the Science of Light - Pr. Schmauss Associated Group}{2019}{Erlangen, Germany}

\cvevent{Study of Multiple Correlated Responses in a Surrogate-based Framework}{Cenaero - Simulation Technologies for Aeronautics}{2018}{Charleroi, Belgium}

\cvevent{Reduction of the Word Equation Problem for a SAT Resolution}{Université Libre de Bruxelles - Computer Sciences Departement}{2017}{Brussels, Belgium}


\clearpage

%% If the NEXT page doesn't start with a \cvsection but you'd
%% still like to add a sidebar, then use this command on THIS
%% page to add it. The optional argument lets you pull up the 
%% sidebar a bit so that it looks aligned with the top of the
%% main column.
% \addnextpagesidebar[-1ex]{page3sidebar}


\end{document}
